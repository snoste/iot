\section{Platform Parameter Design}
\label{sec:platform}

The microkernel slot size C, partition table size P and TDM table size in the control system design play a large role in task timing and scheduling. These parameters' units are shown in Table \ref{tab:design}. The mircokernel slot size will determine the the clock cycles (time) for which the microkernel needs to execute correctly, this was stated to be at least 40.96$\mu$s.

Thus the microkernel slot size could at best reach roughly 40 $\mu$s which would yield to about: $$ C_{min} = 100MHz \cdot 40.96 \mu s = 4096 \;\;Clock \; Cycles$$

This observation was small starting point of educated assumptions about which C and P to chose. Then it was decided rather than picking values for P and C separately to pick a fraction, relating those two. 

In Section \ref{sec:steps} the steps to the findings of sufficient values for P, C and TDM tabel size will be carried out. The goal is to meet the design constraints.

\subsection{Steps of Design Workflow}
\label{sec:steps}
This subsection will discuss selection of P,C, Sensor to actuator delay and some choices considered for poles, TDM table size in order to meet the design constraints. To create an overview of how this was done, a timeline with the steps taken will be presented here.

\begin{enumerate}
	\item First tests made were rather basic and were used to determine the acuteness of poles on the system. A Matlab script was made to
	\item 
	\item
	\item
	\item
	\item
\end{enumerate}

\subsubsection{Sensor to actuator delay}
\label{sec:stad}

\subsection{Implementation of Design steps in Matlab}
\color{red}
Show matlab code for designing and explain matlab approach,
\color{black}