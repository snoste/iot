\section{System Model Derivation and Control Parameter Design}
%System	model	derivation	and	controller	parameter	design
The objectives for the controller is to properly control the given system with a set of design parameters. These parameters, shown in Table \ref{tab:design}, have to be design and put into perspective with our system, which role they play and the best way to derive them adequately.

There are namely a number of steps taken until all parameters can be considered to satisfy the performance constraints of the controller. These basic steps can be seen in the following list.

\begin{enumerate}
	\item Step 1: Derive the system model. Determine the A, B and C matrices in relation to all constants by hand calculations. Inserting this into Matlab. These calculations can be seen in Section \ref{sec:controllerstructure} in full detail.
	\item Step 2: Forming sensible values for C,P TDM\_TABLE\_Size and DELAY. The relationship between these values was inspected in full detail and is discussed in Section \ref{sec:platform}
	\item Step 3: Determine the locations of Ts, Tc and Ta in the TDM table.
	\item Step 4: Determining the sensor to actuator delay in relation to the chosen design parameters, especially P, C and TDM table size is calculated and discussed in Section \ref{sec:stad}. 
	\item Step 5: Simulation for all determined parameters is run and checked if they meet all performance constraints. The check on the performance constraints is also performed during Step 2 by running a script for all parameters and checking every time. This is discussed in more detail in Section \ref{sec:platform}.
	\item Step 6: Finally modifying the Sconfig.c file in the platform to match the design parameters.\color{red} ADD REFERENCE TO SECTION WHEN FINIHSED WRITING ABOUT IT ;) \color{black}
	\item Step 7: Generate both actuator and sensor output files from the platform to feed to Matlab to compare the results with ones from Simulink.\color{red} ADD REFERENCE TO SECTION WHEN FINIHSED WRITING ABOUT IT ;) \color{black}
\end{enumerate}


\begin{table}[h!]
	\centering
	\caption{Design parameters and deliverables for the model. The poles are also something that has to be design or chosen but that will be discussed in Section \ref{sec:controllerstructure}}
	\begin{tabular}{lll}
		\toprule
		Design Parameter & Symbol/Referenced as &Unit\\
		\midrule
		CoMik microkernel slot size& C & Clock Cycles \\
		Partition slot size& P & Clock Cycles\\
		TDM table size& TDM\_TABLE\_SIZE & Natural Number larger than 2\\
		Partition slots allocation	& for Ts, Tc, and Ta & Location in TDM table \\
		Sensor-to-actuator	delay& DELAY & Clock Cycles	\\
		Feedback gain& K & 4x4 matrix \\
		Feedforward gain& F & 4x1 matrix \\
		\midrule		
	\end{tabular}
	\label{tab:design}
\end{table}

\begin{table}[htbp]
	\centering
	\caption{Constants referenced to in calculations}
	\begin{tabular}{llll}
		\toprule
		Symbol & Description & Value & Unit\\ 
		\midrule
		$\theta$ & Masses position  & -&rad \\ 
		$i_m$ & Motor current  & - &A \\ 
		J$_1 = J_2$ & Inertia  & $3.75\cdot10^{-6}$&$Kgm^2$  \\ 
		b & Friction  coefficient   &$ 1\cdot10^{-5} $&Nms/rad\\ 
		k & Torsional spring  & 0.2656 &Nm/rad\\
		d & Torsional damping  & $3.125\cdot10^{-5}$&Nms/rad \\ 
		$K_m$ & Motor constant  & $4.4\cdot10^{-2}$&Nm/A  \\ 
		\midrule
	\end{tabular}
	\label{}
\end{table}

 

\subsection{Controller Structure}
 \label{sec:controllerstructure}
The system can be described with the fourth order differential equation shown in Equation \ref{eq:1} and \ref{eq:2} which will be solved considering Equation \ref{eq:3}. From these equations the Feedback- and Feedforward gain can be determined in relations to the constants in the system.
%\color{red}
%Show some equations 
%\color{black}

\begin{equation}
J_1 \; \ddot{\theta}_1 = K_m \; i_m - k(\theta_1-\theta_2) - d(\dot{\theta}_1-\dot{\theta}_2) - b(\dot{\theta}_1-\dot{\theta}_2)
\label{eq:1}
\end{equation}

\begin{equation}
J_2 \; \ddot{\theta}_2 = -k(\theta_2-\theta_1) - d(\dot{\theta}_2-\dot{\theta}_1) - b(\dot{\theta}_2-\dot{\theta}_1)
\label{eq:2}
\end{equation}

\begin{equation}
\label{eq:3}
\text{\textit{\textbf{x}}}=
\begin{bmatrix}
x_1\\
x_2 \\
x_3 \\
x_4 
\end{bmatrix}
=
\begin{bmatrix}
\theta_1\\
 \theta_2\\
\dot{\theta_1}\\
\dot{\theta_2}
\end{bmatrix}
=
\begin{bmatrix}
\theta_1\\
\theta_2\\
\omega_1\\
\ \omega_2
\end{bmatrix}
\end{equation}

\begin{equation}
\dot{\text{\textit{\textbf{x}}}}(t) = \bar{\text{\textbf{A}}}\text{\textit{\textbf{x}}}(t) +  \bar{\text{\textbf{B}}}\text{\textit{\textbf{u}}}(t) \text{ and } \text{\textit{\textbf{y}}}(t) = \bar{\text{\textbf{C}}}\text{\textit{\textbf{x}}}(t) \qquad \text{ where } \quad \text{\textit{\textbf{u}}}=i_m \text{ and \textit{\textbf{y}}}=\theta_1
\end{equation}

\begin{equation}
\dot{\text{\textit{\textbf{x}}}} = \begin{bmatrix}
& & & \\[0.3em]
\text{\color{white}H\color{black}} &\text{\color{white}h\color{black}} &\text{\color{white}hh\color{black}} &\text{\color{white}hh\color{black}} \\
\text{\color{white}hh\color{black}}& & & \\
\text{\color{white}hh\color{black}} & & &
\end{bmatrix}
\begin{bmatrix}
\theta_1\\
\theta_2\\
\omega_1\\
\omega_2\\
\end{bmatrix}
+
\begin{bmatrix}
\text{\color{white}hh\color{black}}\\
\text{\color{white}hh\color{black}}\\
\text{\color{white}hh\color{black}}\\
\text{\color{white}hh\color{black}}\\
\end{bmatrix}\text{\textit{\textbf{u}}}
\quad \text{ and } \quad
\text{\textit{\textbf{y}}}=\begin{bmatrix}
\text{\color{white}hh\color{black}} &\text{\color{white}hh\color{black}}&\text{\color{white}hh\color{black}}&\text{\color{white}hh\color{black}}
\end{bmatrix}
\begin{bmatrix}
\theta_1\\
\theta_2\\
\omega_1\\
\omega_2\\
\end{bmatrix}
\label{eq:4}
\end{equation}