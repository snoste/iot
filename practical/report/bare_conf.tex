\documentclass[11pt]{article}
\usepackage[pdftex]{graphicx}
\usepackage{amsmath}
 \usepackage[margin=3cm]{geometry}
\usepackage{booktabs}
\usepackage{xcolor}
\usepackage{listings}
\usepackage{multicol}

\usepackage{fancyhdr}

\pagestyle{fancy}
\fancyhf{}
\chead{Practical report Group 14: Broker – 21MN15}
\cfoot{Page \thepage}

%\lstset{basicstyle=\ttfamily,
%	showstringspaces=false,
%	commentstyle=\color{red},
%	keywordstyle=\color{blue}
%}
\usepackage{array}
\newcolumntype{L}[1]{>{\raggedright\let\newline\\\arraybackslash\hspace{0pt}}m{#1}}
\usepackage{biblatex}
\usepackage[colorlinks=true,allcolors=black]{hyperref}
%\usepackage[backend=biber, bibencoding=utf8, style=ieee]{biblatex}
\addbibresource{references.bib}

\begin{document}
\title{Practical Report Group 14: Broker\\ {\fontsize{13}{0}\selectfont Internet of Things (2IMN15) 2016-2017, Eindhoven University of Technology}}

\author{Sai Krishna Kalluri, TU/e, Netherlands, 
		Email: saikrishh.kalluri@gmail.com \& \\  Snorri Stefansson, TU/e, Netherlands, Email: snorriste@gmail.com}
\maketitle



\begin{abstract}
	This practical assignment for the course IoT involves creating a lightning system controlled and managed by different but relevant wireless protocols and separated individual applications. The architecture and development of this system will be the topic of this report.\\	
\end{abstract}
\pagebreak
\tableofcontents

\pagebreak

\section{Group Members}
\begin{table}[htbp]
	\caption{}
	\begin{tabular}{lcllc}
		\toprule
		No & Name & Student ID & Email & Master Program \\ 
		\midrule
		1 & Sai Krishna Kalluri & 1035631 & s.k.kalluri@student.tue.nl & ES \\ 
		2 & Snorri Stefansson & 1033995 & s.stefansson@student.tue.nl & ES \\ 
		\toprule
	\end{tabular}
	\label{}
\end{table}

\section{Group Partners}
\begin{table}[htbp]
	\caption{}
	\centering
	\begin{tabular}{llL{2cm}L{5cm}lc}
		\toprule
		No & Group No  &  Part  & Group Member’s Name and Email & StudID & Program \\ 
		\midrule
		1 & 11 & End Device & \textbf{Christoforos Boukouvalas} \newline c.boukouvalas@student.tue.nl & 0832600 &  \\ 
		&  11& End Device & \textbf{Nikos Avramis} \newline avramisn@gmail.com & 1115964 & AT \\ 
		\midrule
		2 & 10 & Cloud & \textbf{Michalis Kapsalakis}\newline m.kapsalakis@student.tue.nl  & 0979557” & IST \\ 
		& 10 & Cloud & \textbf{Tho Le Phuoc} \newline thole020287@gmail.com &  & CS \\ 
		\toprule
	\end{tabular}
	\label{}
\end{table}



\section{System Description - Understanding the application}

Creating a lighting system with an IoT architecture may sound like an overkill, yet when the application and architecture is designed carefully it becomes essential in everyday life. This system will be described as it will be tested in the final plug fest of this course. The light system is composed of; four lights, each with a user, a user application with a UI, a sensor to detect the users presence, a manager, his web manager UI. These are the most important physical components which allow for a complete system to be operated. 



\section{System Interfaces}
There are three groups which develop this system as stated in Figure \ref{fig:fig1}.
Each team has their purpose. This is the report of the broker, thus will be written with that perspective. Two other teams complete the total assignment, they are cloud group and end-device group and will be references as that through out the report. To shortly describe the relationship with these groups and their purpose; The cloud group is basically responsible of bookkeeping of all information important for the system and the end-device group is responsible of maintaining lights and sensors in the systems which define end device. The end device will present light upon request of the user and be able to detect a user with a camera, sensor. The exact purpose of each team will also come into the protocol and information data exchange related to the modules such as Leshan and MQTT.
\begin{figure}[h]
	\begin{center}
		\includegraphics[width=1\linewidth]{img/design}
		\caption{This Figure is from the lecture slides on IoT \cite{slides}}
		\label{fig:fig1}
	\end{center}
\end{figure}


\section{Implementation, Architecture and Protocols}

There are four significant deliverables for the broker group. Each of them 
\begin{enumerate}
	\item There is a user app for the user to control the light/s, which was selected to be a Android app for this showcase.
	\item mDNS discovery service. Avahi was chosen for this job and is built into linux or can be used with python for integration into a python project.
	\item LWM2M (Rest API) - a 'light weight machine to machine communication' with end device as well as maintaining a CoAP/HTTP server. Leshan was chosen for this task and is written in Java.
	\item MQTT broker which allows lights and sensors to communicate basic commands between one and other with this well suited subscribe and publish service. Mosquitto version 1.4.8 was used for this and installed on linux.
\end{enumerate}

\textit{A laptop with Ubuntu 16.04 64-bit was used to run all these modules except the Android app which was run with an emulator and designed on a Windows computer.}

The architecture is somewhat predefined but many design options were left up to the teams. All communication and information will pass through the broker on its way to the endpoint.

\begin{figure}[h]
	\begin{center}
		\includegraphics[width=0.6\linewidth]{img/overview}
		\caption{}
		\label{fig:fig2}
	\end{center}
\end{figure}

\subsection{LWM2M: CoAP and HTTP with Leshan}


LWM2M is the fundamental ingredient in the broker. It provides a REST API for both CoAP and HTTP which can both be modified to serve the needs of the system. Leshan operates with a server/client infrastructure. Thus the broker runs a server module and other devices connecting to it use a client module. The android app and the cloud service take advantage of the HTTP protocol and the end-device uses the CoAP protocol.

\begin{figure}[h!]
	\centering
	\begin{center}
		\includegraphics[width=1\linewidth]{img/LeshanArc}
		\caption{Architecture of Leshan Server used in implementation} 
		\label{fig:fig3}
	\end{center}
\end{figure}

Event Servlet Functions:Listening to new registrations using RegistrationListener component. Whenever a new registration happens location from database is automatically written into the client based on the order of registration.The location is also updated by the cloud service later if needed. 

Updating the database whenever a value change is detected using ObservationListener component. Observation requests for UserId(Sensor),SenosrState,BehaviourDeployment,Light State, Light Color are requested during registration. A database is created (series of text files) to record and update the Light settings for different users. Whenever a value change is detected in any of these the 'new value' function in the Observation Listener gets activated and the values are updated into database. Inside this function a routine is also made to write the appropriate light values and user information to the light device from database based on the type of behaviour deployment,SensorState and the userId(Sensor).


ClientServletFunctions$(http://serverAddress/api/clients/*)$:\\
Handles all the requests from the cloud service and responds with appropriate responses.Typical requests include reading and writing the resource values using endpointname/Objid/instnId/rsrcId path in the URL.It also handles the handles read and write requests from User App using the same protocol.


SecurityServletFunctions$(http://server_address/api/clients/*)$:\\
Handles storing information of user details in the database  when sent by cloud service and handles authentication of the user from UserApp using these details.

ObjectSpecServlet$(http://server_address/objectspecs/*)$:\\
Handles requests for getting information on objects supported and corresponding details for implementation.


ClientRegistry:Stores the information of registered Clients.

ObservationRegistry:Stores the information on Observation requests created.

LWM2M Request Sender: Responsible for sending request objects(Read,Write,Observe) to the clients.

COAP server: Server for handling COAP messages from clients.

Jetty HTTP Server: HTTP server for handling http requests from USERAPP and Cloud service.

FILES:Files for storing values of different observations(database) and files for creating and updating the Leshan Server page. 

\subsection{User app: Android} 
User App is used to update the light settings of the the user. The settings can be accessed only after logging in with the userId and password. The details are sent to the broker using HTTP protocol(GET). The broker receives the information of all the userID's and their corresponding passwords from cloud service during the "Set User Account" phase. While the intended application is to get the change the corresponding user light settings when logged in, due to a misunderstanding we implemented an which could modify the current light settings of any client active at that moment. Hence any user when logged in with correct details can modify the light settings(provided the client has  a light device) of all the clients active at that point of time.

\begin{figure}[h!]
	\begin{center}
		\includegraphics[width=.4\linewidth]{img/android1}
		\includegraphics[width=.4\linewidth]{img/androidapp}
		\caption{Client Details in android App and Updating Light Values from User App}
		\label{fig:fig3}
	\end{center}
\end{figure}

Note:During the interactions between user app and the broker the IP address of broker is manually set during the initialization phase of the app. No service discovery is used in this case.

\subsection{mDNS: Avahi broker}
mDNS stands for Multicast Domain Name System and it can inform other devices on the same network of the IP of the host. It uses the same structure as DNS but it is a zero configuration service. Avahi implements this zero configuration network discovery profile. On linux Avahi is already installed an requires no extra installation. Both the cloud and end device need to take advantage of this service and can retrieve the IP address with discovery, thus then connecting to the remaining system such a MQTT or Leshan. Figure \ref{fig:avahi} shows how this can be done.

\begin{figure}[h]
	\begin{center}
		\includegraphics[width=1.1\linewidth]{img/avahi}
		\caption{The bash command for running the avahi client on linux}
		\label{fig:avahi}
	\end{center}
\end{figure}


For embedding this into Java, Python or any other high level programming language, it is possible to call the bash command with a 'OS' library and read the output. Setting this up on the broker only required running on startup. Problems during the development were mostly related to port and firewall issues with the laptop or network connection.

\subsection{MQTT: Mosquitto}
MQTT stands for MQ Telemetry Transport and is a very light weight publish and subscribe protocol. The Mosquitto broker is installed on the linux machine and then the clients can both publish messages to topics and subscribe to the topics as well. This module worked seamlessly after install. In the testing section it can be observed how messages are published (this is only for making sure from the broker side that the MQTT broker is working).

\section{Product Testing}

From day one all code had to be tested frequently and thoroughly, eliminating errors and all bad test cases. Most testing had to be done with the Leshan server. A dummy client was created to simulated the end device and then HTTP request where sent from browser and tested with the android app as well.

\subsection{MQTT testing}

Testing was done on MQTT for making sure that clients could connect to it and communicate the FREE/OCCUPIED between the sensor and light device. That part worked for all teams connecting to the system. As can be seen in Figure \ref{fig:mqtt3}.
\begin{figure}[h]
	\begin{center}
		\includegraphics[width=\linewidth]{img/mqtt3}
		\caption{Mosquitto tested during Plugfest with other teams. MQTT broker works and clients can communicate.}
		\label{fig:mqtt3}
	\end{center}
\end{figure}

\subsection{Leshan Testing}

This section will show testing of Leshan, how it responded to requests and how the development interface looks like. In Figure \ref{fig:observe} the log for the Leshan server is displayed when an request from the web app is made to observe a sensor value (resource 10350).

\begin{figure}[h]
	\begin{center}
		\includegraphics[width=\linewidth]{img/screenshot-observe}
		\caption{It can be seen here that an observation was made on an resource}
		\label{fig:observe}
	\end{center}
\end{figure}

An important part to be working is the updating of ownership and thus it is useful to show the testing of the part. In Figure \ref{fig:own} the Leshan server log can be seen when the Cloud group updates the ownership via their manager web app seen in Figure \ref{fig:cloudown}.

\begin{figure}[h]
	\begin{center}
		\includegraphics[width=1.18\linewidth]{img/screenshot-ownership}
		\caption{Here it can be seen that the ownership was updated from another IP address, namely the Cloud}
		\label{fig:own}
	\end{center}
\end{figure}

\begin{figure}[h]
	\begin{center}
		\includegraphics[width=.8\linewidth]{img/screenshot-cloud-ownership}
		\caption{The Cloud group updates the ownership in the Leshan server via their web app in this picture. The IP matching the IP in the logs in Figure \ref{fig:own}}
		\label{fig:cloudown}
	\end{center}
\end{figure}

\subsection{App Testing}
In Figure \ref{fig:app} the app can be seen working with multiple clients. The light can be controlled as indicated.

\subsection{Avahi Testing}

Avahi was tested multiple times with the red team. Figure \ref{fig:avahi} shows the command publish the service and then if the client uses the same parameters it will connect. 
\section{Discussion and results}

Development of this practical project of the course introduced how a IoT solution would be developed from scratch. Starting with only basic code samples and working towards a functional automated lightning settings. The architecture was chosen to favor the knowledge of project participants to be able to finish in time. That comes to show that with any platform and operating systems used, the protocols will keep their structure, as indented. 

To conclude the broker part was launched from a ubuntu laptop that was able to run all required programs. The Mosquitto broker, mDNS service and the Leshan server. The android app is run on an emulator in Windows. Both Mosquitto and mDNS were started from the terminal directly with as detailed debug messages as possible to observe all traffic and clients using them. The Leshan server was as previously mentioned run from eclipse or as a *.jar executable. This completes the system of the broker part and its deliverables defined by the instructions.
\section{Evaluation}

Now the systems has been developed so that it can talk to any end-device with the predefined information sharing. The broker can be turned on, make it self discoverable to any end-device (mDNS), allow for communication between light and sensor device (MQTT) and finally allows for end-devices to register as clients in Leshan and be accessed by the cloud with jetty inside Leshan. This is the complete, simplified, architecture use case.	
\section{Reflection}

Lessons learned during this practical assignment is definitely practical, where using many different protocols actually used in IoT to create small smart network of sensors and lights.

Working with the other groups was interesting as none of the groups knew each other which required each group to understand how much the other groups had done of programming and practical work such as this. The end device group had less experience than the cloud group which resulted in a small delay in the development process where the system should actually work as specified. There was one hick-up, that the cloud team expected that the end-device would register one object with both light and sensor information in it. This is not a problem with the broker but just how the cloud reads the information from Leshan. This was the last problem encounterd, during the plug-fest.

All in all the group work went very well and everybody put all their resources into the project. Which was a good experience.   
\section{Contribution}
\subsection{UserApp(Android)}
begin{enumerate}
\item LoginActivity: Validating the entered user details against server- Snorri.
\item ClientActivity: Collecting information related to current state of the system from broker- SaiKrishna.
\item ControlActivity: Creating User interface for light objects-Snorri
\item  ControlActivity: Communicating the updated light state to the broker- Sai Krishna.
\item Utilities:Creating and handling HTTP requests,parsing JSON information from server-Sai Krishna.
end{enumerate}

\subsection{Server(Leshan)}
begin{enumerate}
\item ClientServlet:Handling read, write and observation requests from cloud and UserApp-Snorri
\item EventServlet:Performing operations during registrations, creating new observation requests- Sai Krishna

\item EventServlet(CentralizedDeployment):Handling new observations,updating databases,centralized Deployment routine-Snorri

\item Security Servlet:Storing User details received from cloud and handling Login validation requests from UserAPP- Sai Krishna.

\item Utilities:Json parsing,Dummy Clients for testing-Sai Krishna

\subsection{Avahi}-Creating network discovery for both clients and cloud service-Snorri.
\subsection{MQTT}-Creatig MQTT broker client publish and subscriptions -Snorri.

\subsection{Report} Report for the project-Snorri and Sai Krishna.

\printbibliography

\end{document}